
\chapter{使用简介}
\label{chap:guide}

为方便使用及更好的展示\LaTeX{}排版的优秀特性,本人对模板的框架和文件体系进行了一些处理,尽可能地对各个功能和板块进行了模块化和封装,对于初学者来说,众多的文件目录也许会让人觉得有些无所适从,但阅读完下面的使用说明后,您会发现原来使用思路是简单而清晰的,而且,当对\LaTeX{} 有一定的认识和了解后,会发现其相对Word类排版系统的极具吸引力的优秀特性。所以,如果您是初学者,请不要退缩,请稍加尝试和坚持,让自己领略到\LaTeX{}的非凡魅力。

\section{先试试效果}

\texttt{hfutThesis}模板不仅只是提供了相应的类文件,同时也提供了包括参考文献等在内的完成学位论文的一切要素,所以,下载时,推荐下载整个ucasthesis文件夹,而不是单独的文档类。

在下载\texttt{hfutThesis}宏包之前,首先检查工作电脑是否满足第~\ref{sec:sysRequire}节中对系统的要求,主要是对\TeX{}安装包的要求。若是第一次安装使用,推荐首先在电脑上运行一个简单的中文教程,既是对安装包的确认,也是帮助认识\LaTeX{}的工作流程。下载\texttt{hfutThesis}文件夹后,请在文件夹目录下,按照摘要或者README.md中的基本流程首次运行宏包,确认系统工作状态良好。

编译完成,若工作正常将得到对应的PDF文档myThesis.pdf。恭喜,至此你就完成了一次简单的系统运行!后续,你就可以参见各子目录内容进行对应内容的添加。如若编译失败,仔细查看log文件,找出对应的问题(应该大多是系统配置问题)。

为了与合肥工业大学研究生院的MS Word$^{\circledR}$版对照,需将合肥工业大学研究生院提供的Word版转化为PDF文件(合肥工业大学学位论文撰写规范.pdf),对比编译出来的PDF文件和官网生成模版。

\section{范例文档结构}
\label{sec:example}

HowToUseIt.pdf是该项目最新编译生成的模版文件,目的是帮助用户即使在不了解\LaTeX{}排版技术的情况下能够快速的入门,并利用该项目的\texttt{hfutThesis}模板加快学位论文的排版工作,从而能够将更多的时间专注于论文所研究的问题上,希望能够达到这样的一点点帮助!

范例文档的结构在第~\ref{sec:howtouse}中已经有所提及,这里只是单纯介绍一下正文中对使用方法的描述部分。正文部分包含四章:引言、使用简介、文本结构和公式图表使用。前两章较多的项目和使用方法的介绍,后两章着重对\LaTeX{}排版技术在学位论文中的常见使用。用户可以针对性的进行阅读,也可以直接跳过,把范例中的内容作为当作手册使用。




